\chapter{绪论}
\chaptermark{苹果基因组可视化方法研究}
	\section{苹果基因组可视化背景}
	数据可视化主要旨在借助于图形化手段,清晰有效地传达与沟通信息。但是,这并不就意味着,数据可视化就一定因为要实现其功能用途而令人感到枯燥乏味,或者是为了看上去绚丽多彩而显得极端复杂。为了有效地传达思想概念,美学形式与功能需要齐头并进,通过直观地传达关键的方面与特征,从而实现对于相当稀疏而又复杂的数据集的深入洞察。然而,设计人员往往并不能很好地把握设计与功能之间的平衡,从而创造出华而不实的数据可视化形式,无法达到其主要目的,也就是传达与沟通信息。数据可视化与信息图形、信息可视化、科学可视化以及统计图形密切相关。当前,在研究、教学和开发领域,数据可视化乃是一个极为活跃而又关键的方面。“数据可视化”这条术语实现了成熟的科学可视化领域与较年轻的信息可视化领域的统。
	\section{苹果基因组可视化的目的和意义}
		\subsection{名词解释}
		数据可视化主要旨在借助于图形化手段,清晰有效地传达与沟通信息。但是,这并不就意味着,数据可视化就一定因为要实现其功能用途而令人感到枯燥乏味,或者是为了看上去绚丽多彩而显得极端复杂。为了有效地传达思想概念,美学形式与功能需要齐头并进,通过直观地传达关键的方面与特征,从而实现对于相当稀疏而又复杂的数据集的深入洞察。然而,设计人员往往并不能很好地把握设计与功能之间的平衡,从而创造出华而不实的数据可视化形式,无法达到其主要目的,也就是传达与沟通信息。数据可视化与信息图形、信息可视化、科学可视化以及统计图形密切相关。当前,在研究、教学和开发领域,数据可视化乃是一个极为活跃而又关键的方面。“数据可视化”这条术语实现了成熟的科学可视化领域与较年轻的信息可视化领域的统
		\subsection{系统特色}
		数据可视化主要旨在借助于图形化手段,清晰有效地传达与沟通信息。但是,这并不就意味着,数据可视化就一定因为要实现其功能用途而令人感到枯燥乏味,或者是为了看上去绚丽多彩而显得极端复杂。为了有效地传达思想概念,美学形式与功能需要齐头并进,通过直观地传达关键的方面与特征,从而实现对于相当稀疏而又复杂的数据集的深入洞察。然而,设计人员往往并不能很好地把握设计与功能之间的平衡,从而创造出华而不实的数据可视化形式,无法达到其主要目的,也就是传达与沟通信息。数据可视化与信息图形、信息可视化、科学可视化以及统计图形密切相关。当前,在研究、教学和开发领域,数据可视化乃是一个极为活跃而又关键的方面。“数据可视化”这条术语实现了成熟的科学可视化领域与较年轻的信息可视化领域的统
		
		数据可视化主要旨在借助于图形化手段,清晰有效地传达与沟通信息。但是,这并不就意味着,数据可视化就一定因为要实现其功能用途而令人感到枯燥乏味,或者是为了看上去绚丽多彩而显得极端复杂。为了有效地传达思想概念,美学形式与功能需要齐头并进,通过直观地传达关键的方面与特征,从而实现对于相当稀疏而又复杂的数据集的深入洞察。然而,设计人员往往并不能很好地把握设计与功能之间的平衡,从而创造出华而不实的数据可视化形式,无法达到其主要目的,也就是传达与沟通信息。数据可视化与信息图形、信息可视化、科学可视化以及统计图形密切相关。当前,在研究、教学和开发领域,数据可视化乃是一个极为活跃而又关键的方面。“数据可视化”这条术语实现了成熟的科学可视化领域与较年轻的信息可视化领域的统
		
		数据可视化主要旨在借助于图形化手段,清晰有效地传达与沟通信息。但是,这并不就意味着,数据可视化就一定因为要实现其功能用途而令人感到枯燥乏味,或者是为了看上去绚丽多彩而显得极端复杂。为了有效地传达思想概念,美学形式与功能需要齐头并进,通过直观地传达关键的方面与特征,从而实现对于相当稀疏而又复杂的数据集的深入洞察。然而,设计人员往往并不能很好地把握设计与功能之间的平衡,从而创造出华而不实的数据可视化形式,无法达到其主要目的,也就是传达与沟通信息。数据可视化与信息图形、信息可视化、科学可视化以及统计图形密切相关。当前,在研究、教学和开发领域,数据可视化乃是一个极为活跃而又关键的方面。“数据可视化”这条术语实现了成熟的科学可视化领域与较年轻的信息可视化领域的统
	\section{论文组织结构}
	数据可视化主要旨在借助于图形化手段,清晰有效地传达与沟通信息。但是,这并不就意味着,数据可视化就一定因为要实现其功能用途而令人感到枯燥乏味,或者是为了看上去绚丽多彩而显得极端复杂。为了有效地传达思想概念,美学形式与功能需要齐头并进,通过直观地传达关键的方面与特征,从而实现对于相当稀疏而又复杂的数据集的深入洞察。然而,设计人员往往并不能很好地把握设计与功能之间的平衡,从而创造出华而不实的数据可视化形式,无法达到其主要目的,也就是传达与沟通信息。数据可视化与信息图形、信息可视化、科学可视化以及统计图形密切相关。当前,在研究、教学和开发领域,数据可视化乃是一个极为活跃而又关键的方面。“数据可视化”这条术语实现了成熟的科学可视化领域与较年轻的信息可视化领域的统
	
	数据可视化主要旨在借助于图形化手段,清晰有效地传达与沟通信息。但是,这并不就意味着,数据可视化就一定因为要实现其功能用途而令人感到枯燥乏味,或者是为了看上去绚丽多彩而显得极端复杂。为了有效地传达思想概念,美学形式与功能需要齐头并进,通过直观地传达关键的方面与特征,从而实现对于相当稀疏而又复杂的数据集的深入洞察。然而,设计人员往往并不能很好地把握设计与功能之间的平衡,从而创造出华而不实的数据可视化形式,无法达到其主要目的,也就是传达与沟通信息。数据可视化与信息图形、信息可视化、科学可视化以及统计图形密切相关。当前,在研究、教学和开发领域,数据可视化乃是一个极为活跃而又关键的方面。“数据可视化”这条术语实现了成熟的科学可视化领域与较年轻的信息可视化领域的统
	
	数据可视化主要旨在借助于图形化手段,清晰有效地传达与沟通信息。但是,这并不就意味着,数据可视化就一定因为要实现其功能用途而令人感到枯燥乏味,或者是为了看上去绚丽多彩而显得极端复杂。为了有效地传达思想概念,美学形式与功能需要齐头并进,通过直观地传达关键的方面与特征,从而实现对于相当稀疏而又复杂的数据集的深入洞察。然而,设计人员往往并不能很好地把握设计与功能之间的平衡,从而创造出华而不实的数据可视化形式,无法达到其主要目的,也就是传达与沟通信息。数据可视化与信息图形、信息可视化、科学可视化以及统计图形密切相关。当前,在研究、教学和开发领域,数据可视化乃是一个极为活跃而又关键的方面。“数据可视化”这条术语实现了成熟的科学可视化领域与较年轻的信息可视化领域的统
	
	数据可视化主要旨在借助于图形化手段,清晰有效地传达与沟通信息。但是,这并不就意味着,数据可视化就一定因为要实现其功能用途而令人感到枯燥乏味,或者是为了看上去绚丽多彩而显得极端复杂。为了有效地传达思想概念,美学形式与功能需要齐头并进,通过直观地传达关键的方面与特征,从而实现对于相当稀疏而又复杂的数据集的深入洞察。然而,设计人员往往并不能很好地把握设计与功能之间的平衡,从而创造出华而不实的数据可视化形式,无法达到其主要目的,也就是传达与沟通信息。数据可视化与信息图形、信息可视化、科学可视化以及统计图形密切相关。当前,在研究、教学和开发领域,数据可视化乃是一个极为活跃而又关键的方面。“数据可视化”这条术语实现了成熟的科学可视化领域与较年轻的信息可视化领域的统
	
	数据可视化主要旨在借助于图形化手段,清晰有效地传达与沟通信息。但是,这并不就意味着,数据可视化就一定因为要实现其功能用途而令人感到枯燥乏味,或者是为了看上去绚丽多彩而显得极端复杂。为了有效地传达思想概念,美学形式与功能需要齐头并进,通过直观地传达关键的方面与特征,从而实现对于相当稀疏而又复杂的数据集的深入洞察。然而,设计人员往往并不能很好地把握设计与功能之间的平衡,从而创造出华而不实的数据可视化形式,无法达到其主要目的,也就是传达与沟通信息。数据可视化与信息图形、信息可视化、科学可视化以及统计图形密切相关。当前,在研究、教学和开发领域,数据可视化乃是一个极为活跃而又关键的方面。“数据可视化”这条术语实现了成熟的科学可视化领域与较年轻的信息可视化领域的统
	
	数据可视化主要旨在借助于图形化手段,清晰有效地传达与沟通信息。但是,这并不就意味着,数据可视化就一定因为要实现其功能用途而令人感到枯燥乏味,或者是为了看上去绚丽多彩而显得极端复杂。为了有效地传达思想概念,美学形式与功能需要齐头并进,通过直观地传达关键的方面与特征,从而实现对于相当稀疏而又复杂的数据集的深入洞察。然而,设计人员往往并不能很好地把握设计与功能之间的平衡,从而创造出华而不实的数据可视化形式,无法达到其主要目的,也就是传达与沟通信息。数据可视化与信息图形、信息可视化、科学可视化以及统计图形密切相关。当前,在研究、教学和开发领域,数据可视化乃是一个极为活跃而又关键的方面。“数据可视化”这条术语实现了成熟的科学可视化领域与较年轻的信息可视化领域的统
	
	数据可视化主要旨在借助于图形化手段,清晰有效地传达与沟通信息。但是,这并不就意味着,数据可视化就一定因为要实现其功能用途而令人感到枯燥乏味,或者是为了看上去绚丽多彩而显得极端复杂。为了有效地传达思想概念,美学形式与功能需要齐头并进,通过直观地传达关键的方面与特征,从而实现对于相当稀疏而又复杂的数据集的深入洞察。然而,设计人员往往并不能很好地把握设计与功能之间的平衡,从而创造出华而不实的数据可视化形式,无法达到其主要目的,也就是传达与沟通信息。数据可视化与信息图形、信息可视化、科学可视化以及统计图形密切相关。当前,在研究、教学和开发领域,数据可视化乃是一个极为活跃而又关键的方面。“数据可视化”这条术语实现了成熟的科学可视化领域与较年轻的信息可视化领域的统