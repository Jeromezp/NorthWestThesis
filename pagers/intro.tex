\chapter{绪论}
\chaptermark{绪论}
	\section{研究背景}
	随着第二代高通量的测序技术的发展,测序通量在以超过摩尔定律增长趋势快速增长,而成本却直线下降,这无疑对科学家通过分子水平开展科学研究提供了一个最有力的支持和帮助。但海量的数据对从事生物信息分析的人员却提出了巨大的挑战,如何及时、高效并准确的处理和分析这些数据,也是生物信息工作者开口必谈的话题~\cite{夏艳2013水稻比较基因组和进化生物学数据库的构建研究}。\\
	\indent 当今大量的核酸/ 蛋白质序列、基因/ 蛋白质结构和功能数据 , 各种疾病相关数据以及生物文献数据等正飞速增长。如何充分利用这些数据、挖掘数据间潜在生物关系、解释这些数据 , 是计算机和生物学家面临的巨大挑战。大多数生物学知识既不能象物理学那样以数学公式表示 , 又不能象计算机学那样以逻辑公式表示 , 但却能以表格、图形、网络等直观的形式体现 , 因而可借助于科学可视化技术进行生物数据挖掘。\\
	\indent 在生物信息数据分析的众多过程中,如序列拼接、序列比对、SNP检查、表达量分析 等,都出现了一些专业的自动化的软件工具,用于帮助研究人员进行数据分析,从而大大提高了数据分析的效率,但是数据分析结果的正确有效性,仍依赖于研究人员的人工参与,而面对杂乱无章的数据文件时,人工参与的效率往往较低相比之下,图形或图表能很直观的表示数据特征,更易于被人阅读和理解,如果能将分析结果数据以图形或图表的方式进行,并提供一些交互性操作界面,将会极大的提高数据分析效率。 因此可视化工具成为成为目前基因组发展和研究的重要手段和工具。
		\subsection{基因组可视化}
		基因组可视化 属于 “数据可视化”的一种,相当于数据可视化理论方法在基因组学数据上的具体应用。由于基因组数据的独特性,因此又有着一些独特的可视化方法产生。
		基因组可视化,即对基因组数据进行可视化,通常指对最基本的基因组DNA序列,和注释数据等基因组相关的分析数据,按照一定的用户友好方式,使用图形元素表达出来,方便视角直观地识别已知或未知的数据模式,或者比较差异等。
		\subsection{苹果基因组可视化}
		苹果作为悠久栽培历史的果树树种,在世界上产量排名第4。对于苹果基因组的研究一直是科研人员热门研究对象,随着基因组织学的发展,包括以全基因测序为目标的结构基因组织学和以基因功能鉴定为目标组织学在不断发展,通过对苹果基因组的可视化研究,为科研工作者深入了解苹果基因组,认识基因性状之间的联系提供了可靠地方式。
	\section{研究的目标和意义}
		\subsection{研究目标}
		本文以苹果基因组为研究对象,选取蔷薇科其它同源物种及拟南芥物种基因组进行研究对比,以基因组可视化工具为研究对象,在可视化工具原理分析与应用、基因组数据处理和转储、服务端平台搭建和维护等方面做了一系列工作,具有交叉学科及知识跨度广的的特点。本通过查阅相关文献及资料,选择GMOD通用基因模型数据库中的可视化工具集GBrowse作为基因可视化工具,对苹果基因组中的染色体,CDS,contig等基因序列进行可视化处理,同时以苹果基因序列作为对比对象对拟南芥及苹果的同源物中相似基因序列进行可视化,实现对苹果基因组可视化及具有相同基因序列的同源物种进行可视化处理。
		\subsection{研究意义}
		苹果作为悠久栽培历史的果树树种,在世界上产量排名第4。对于苹果基因组的研究一直是科研人员热门研究对象,随着基因组织学的发展,包括以全基因测序为目标的结构基因组织学和以基因功能鉴定为目标组织学在不断发展,通过对苹果基因组的可视化研究,为科研工作者深入了解苹果基因组,认识基因性状之间的联系提供了可靠地方式,同时为解决如何确定大量基因序列功能问题及通过对比同源产物基因组确定相应的功能区域提供了更加有效的方法,为研究苹果基因组提供了可靠友好的方式。
	\section{基因组可视化工具}
		\subsection{基因组可视化工具发展}
		随着海量的基因序列的爆炸式增长,发展基因序列有效的可视化方法成为当今热门的话题,同时支持生物大数据的深度分析、集成、研究和服务,已经成为基因组学研究领域面临的一项重要课题。目前国内外的研究机构和公司开发了许多个基于web技术的基因组浏览器,以满足基因组可视化、大规模基因组数据分析和应用需求。根据基因组浏览器的使用形式,可以分为基于桌面的浏览器和web 浏览器两种。
		\subsection{基于web基因组可视化工具}
		根据基因组浏览器的使用形式,可以分为基于桌面的浏览器和web 浏览器两种。基于桌面的浏览器虽然方便加载本地数据,但是由于基因组数据十分庞大,所以对PC 的要求比较高,而且需要在本地安装客户端程序。 基于 web 的基因组浏览器不用安装,无需进行繁琐的软件配置,用户只需要通过网页浏览器连接到 Internet 就可以使用,对用户个人 PC 的要求不高,而且由于数据存储在服务器端,所以用户本地电脑不用消耗太多存储空间来贮放数据。相比较而言,基于 web 的基因组浏览器更有发展前景。目前来说比较主流的web端基因组浏览器有,GBrowse,JBrowse,UCSC Genome Browser等一系列开源工具。
	\section{本文研究内容}
	本文研究的内容围绕着苹果基因组及其同源物基因组可视化问题, 采用目前主流的web基因组浏览器GBrowse,JBrowse,UCSC Genome Browser进行对比展示;同时通过将苹果基因组及其同源物基因组数据进行合理化处理,在合理可视化同时考虑到减少服务器压力采用MySQL数据对数据进行整合;通过合理配置Apache相关配置选项完成正真意义上的web端可视化效果。 具体工作包含以下方面:
	\begin{enumerate}
		\item Linux环境选取及适配多种类型的Linux操作系统。 
		\item  Apache、MySQL的源码编译安装及配置,解决模块依赖问题。
		\item  GBrowse工具集安装及配置,解决与MySQL、Apache等关联问题。
		\item  JBrowse,UCSC Genome Browser的源码编译安装及配置,解决模块依赖问题。
		\item 苹果及其同源物种基因组数据集处理,及对相关数据及的整合调试。
		\item 对比不同web基因组浏览器的可视化效果,并对每种浏览器进行深入解析。
	\end{enumerate}
	
	\section{本文内容安排}
	本文第 1 章是绪论, 分为研究背景, 从基因组可视化发展到苹果基因组可视化的发展进行介绍; 研究的目标和意义, 主要说明了本文要达到的目标和本文研究的意义; 基因组可视化工具, 即从pc端到web端浏览器进行介绍说明目前主流的基因组浏览器发展趋势;研究的内容和本文的内容安排,提纲挈领地介绍了本文的内容和文章内容安排。\\
	\indent 本文的第 2章是基于web基因组可视化工具的概述,主要介绍了GBrowse,JBrowse,UCSC Genome Browser 的系统架构及运行原理,通过深入的分析和对比不同浏览器之间的差异及优势。\\
	\indent 本文的第 3 章是基于web基因组浏览器系统搭建。主要介绍了GBrowse,JBrowse,UCSC Genome Browser工具集在Linux平台下搭建,并通过安装配置Apache及MySQL等web服务软实现可以通过IP进行远程访问。\\
	\indent 本文的第 4 章是数据集的处理。 主要介绍了针对GBrowse对数据格式的需求对数据进行进一步处理并针对数据量大,缓解服务端压力的思想对数据进行转储。\\
	\indent 本文的第 5 章是苹果基因与蔷薇科植物基因组可视化验证。 主要在GBrowse基因组浏览器中对比苹果基因组与拟南芥,其他蔷薇科植物基因组进行可视化验证。\\
	\indent 本文的第 6 章是总结与展望。主要对研究的总结及对今后的研究进行展望。\\
	\indent 其余部分包括目录, 致谢和参考文献等。