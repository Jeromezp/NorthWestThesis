%% 中文摘要
\chapter*{苹果基因组可视化方法研究}

\vspace{1em}
{\large {\heiti 摘要: }}\normalsize{\songti 
随着基因组测序技术的发展和测序成本的降低,各物种的海量基因组数据不断涌现。有效查看数据就显得十分重要。基因组可视化 属于 “数据可视化”的一种,是数据可视化理论方法在基因组学数据上的具体应用。同时,由于基因组数据的特殊性(例如基因组有若干带有坐标位置的基因片段组成),一些针对性的可视化方法随之产生。其中,Web端的基因组可视化工具逐渐受到大量科研人员的青睐。GBrowse作为Web端基因数据可视化工具,能够对目前已知的绝大部分生物数据进行可视化展示,并且其具有强大的可扩展性,对于以大量字符串所表示的生物学数据具有非常明显的优势。本文在对主流的Web端基因组浏览器GBrowse,JBrowse,UCSC Genome Browser进行分析对比的基础上,以GBrowse为主要工具对我校的特色农作物苹果的基因组进行了可视化分析。为了对苹果的基因功能提供参考,本文还对研究比较透彻的与苹果同源的其它蔷薇科物种(拟南芥、桃)基因组进行了可视化,以友好的方式显示给研究人员,方便他们进行比较分析。最后,这些可视化的结果都呈现在云服务器网站:123.206.61.168上,访问速度较国外同类网站有明显的改善。从而有助于研究者较为透彻辨析苹果基因组同源物基因序列的差异性和相似性,对基因研究人员有着实际应用价值。苹果基因组可视化网站的建设为研究人员直观、快速地定位、查询感兴趣的基因提供了便利。
}

{\large {\heiti 关键词:}}\normalsize{基因组可视化;Web端可视化工具;GBrowse;JBrowse;UCSC Genome Browser}
\thispagestyle{empty}