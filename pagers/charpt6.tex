\chapter{总结与展望}
	\chaptermark{总结与展望}
	\section{总结}
	本文使用GBrowse作为核心可视化工具, 实现了苹果基因组可视化及其同源物红桃及拟南芥基因组的可视化,从中得到如下几点结论:\\
	\indent(1) GBrowse安装使用时需要解决的问题比较繁杂,针对Perl包依赖需要从中一次找出依赖文件分别进行安装测试。同时在不同系列的Linux环境下GBrowse依赖关系存在差异性,解决时需要针对不同的情况进行逐一排查和对比。在安装时针对Apache配置问题,对于CGI的配置文件权限问题,文件权限问题,数据格式问题都需要通过。\\
	\indent(2)GBrowse的使用者可以上传相应的数据文件通苹果基因进行对比可视化。\\
	\indent	(3) 通过对苹果基因组及其同源物和其他物种进行可视化对比,可以是用户更加方便直观的观察到苹果基因组的功能。\\
	\section{展望}
	由于时间所限, 本文只针对GBrowse在可视化苹果基因组做了研究,对于其他Web端可视化工具没有进行细致的研究,同时对于GBrowse的代码层的细节没有过于深入的研究,对于Perl的生物扩展包没有进一步的研究。其次,在JBrowse中的框架方面,没有对跨平台框架electron做进一步深入的研究。